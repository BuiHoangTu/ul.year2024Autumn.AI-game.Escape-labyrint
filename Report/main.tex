\documentclass[a4paper,12pt]{article}

% Packages
\usepackage[utf8]{inputenc}
\usepackage{geometry}
\usepackage{graphicx}
\usepackage{hyperref}
\usepackage{amsmath}
\usepackage{listings}
\usepackage{xcolor}

% Page Setup
\geometry{margin=1in}

% Code Snippet Settings
\lstset{
    basicstyle=\ttfamily\small,
    backgroundcolor=\color{lightgray!20},
    frame=single,
    breaklines=true,
    postbreak=\mbox{\textcolor{red}{$\hookrightarrow$}\space},
    numbers=left,
    numberstyle=\tiny\color{gray},
    keywordstyle=\color{blue},
    commentstyle=\color{green!70!black},
    stringstyle=\color{red},
}

% Title Section
\title{Unity RTS Unit Report: Pathfinding and Finite State Machine
\\ \small Word Count: XXX 
\\ Student ID: 24005665}

\author{Hoang Tu BUI}
\date{\today}

\begin{document}

% Title Page
\maketitle
\tableofcontents
\newpage

% Sections
\section{Introduction}


\subsection{Game Overview}
The game designed and developed for this project is a real-time strategy (RTS) in which two teams: Finder (F) and Escaper (E), compete against each other. 
The Finder team has to capture the Escaper before it reaches one of three exits.
Both teams are controlled by AI agents implemented using Finite State Machines (FSMs) with the support of pathfinding algorithms for navigation.


\subsection{Entities}
The game consists of 4 main entities:
\begin{itemize}
    \item Finder: 
    There are two Finders in the game, each with the primary objective of locating and capturing the Escaper.
    The Finders have superior base speed (10\%) compared to the Escaper allowing them to cover more ground.
    They are also capable of informing each other of the Escaper's location when one spots it.
    Their pursuit is aided by a limited burst of speed (30\%) which while not as fast as the Escaper's speed burst, is enough to help keeping track of the Escaper.
    After the burst wornout for both parties, the Finders are mostly guaranteed to catch the Escaper.
    
    \item Escaper: 
    The Escaper is a single unit attempting to reach one of the three exits on the map.
    It has faster bursts of speed (80\% of its base speed) compared to the Finders, allowing it to evade capture and hide.
    The Escaper also possesses an advantage in terms of vision range (43\% further), allowing it to spot the Finders from a distance.
    
    \item Exit: 
    There are three exits scattered throughout the map. 
    The exits are positioned in such a way that the Finders have no way of covering all of them at once.
    
    \item Obstacle: 
    Static objects that placed around the map limiting vision and movement. 
    They can either be hiding spots or dead end for the Escaper.
\end{itemize}

In summary, this game presents a highly strategic scenario where two Finders work together to capture a lone Seeker, who must utilize speed bursts and environmental knowledge to escape.



\section{Analysis}
In this subsection, I discuss the rationale behind the choice of using the chosen Artificial Intelligence (AI) techniques for both the Escaper and Finder units in the game.
I explain the selection of the Finite State Machine (FSM) to govern the agents' behaviors, analyze its functionality, and delve into the challenges faced during the implementation.


\subsection{Rationale for Pathfinding}
Pathfinding is a crucial aspect of any game where units need to navigate the environment.
For this game, the Escaper needs to find the best path to reach one of the exits while avoiding the Finders and the Finders need to pursuit the Escaper and find the optimal path to guard the exits.
I chose to implement the A* algorithm and BFS.

A* can be considered the best pathfinding algorithm.
It combines elements of both Dijkstra's algorithm and Greedy Best-First Search, using two key functions: the cost to reach the node (g) and the estimated cost to the goal (h). 
The algorithm maintains a priority queue, exploring nodes with the lowest total estimated cost (f = g + h). 
As it progresses, A* chooses the most promising node based on these values, resulting the most efficient path to the goal.

BFS, on the other hand, is a simpler algorithm that explores all neighbor nodes at the present depth prior to moving on to nodes at the next depth level.
This algorithm is used solely for the Finders to navigate while patrolling the exit where they explore the area without a specific goal.


\subsection{Rationale for Finite State Machine}
Finite State Machines (FSMs) are a common method for modeling the behavior of AI agents in games.
The FSM ensures that the agents' behavior is divided into a set of states, each with specific conditions for transitioning between states.
It is a simple and effective way to manage complex behaviors, making it ideal for the game's units, provide clear logic for the agents' decision-making process.
In this game, Finite State Machines are used to model the behavior of both the Escaper, the Finders and the movement of both types of units.

For the Escaper, the FSM is used to model the behavior of the Escaping, which includes the states of:
\begin{itemize}
    \item Idle: The Escaper remains stationary while calculating the next move.
    \item Move to Exit: The Escaper moves towards the safe, nearest exit.
    \item Flee: When the Escaper sees a Finder and the Finder is also looking to the Escaper place.
        It will move backward to the nearest safe spot.
\end{itemize}

For the Finders, the FSM includes the states of:
\begin{itemize}
    \item Idle: The Finder remains stationary while choosing the next spot to patrol and calculating the path.
    \item Patrol: The Finder moves around the exit area.
        After reaching the destination, it will observe the area.
    \item Observe: The Finder will turn around to look for the Escaper.
        If nothing suspicious is found, it will return to the Idle state.
    \item Chasing: At any point, if any Finder sees the Escaper, all Finders will start chasing the Escaper.
        If they lose sight of the Escaper, they will return to the Observe state to see if the Escaper is hiding nearby.
\end{itemize}

Finally, the FSM is used to model the movement of units.
They both share the same sets of movement states:
\begin{itemize}
    \item Idle: The unit remains stationary.
    \item Turn Left: The unit rotates to the left.
    \item Turn Right: The unit rotates to the right.
    \item Forward: The unit moves forward. It is useful for common movement.
    \item Backward: The unit moves backward. It is useful for fleeing where the unit does not have time to turn around.
    \item Burst Forward: The unit moves forward at a faster speed. It is useful for chasing. After the duration, the unit will return to the Forward state.
    \item Burst Backward: The unit moves backward at a faster speed. After the duration, the unit will return to the Backward state.
\end{itemize}
While most of the states are free to transition to any other states, Burst will be transitioned to Forward or Backward when they don't have Burst energy left.


\subsection{Challenging in Implementation}
The main challenge faced during the implementation of the FSM was the complexity of managing the transitions between states.
Some proposed methods like using transition tables not only made the code more bloated but also made development process restrictive.
So my choice was to use a more flexible approach where each state is responsible for deciding the next state based on the current conditions.

Another challenge was the implementation of the Flee state. 
At first, I made the Escaper run away when it sees a Finder.
But this made the character keep running back and forth because the pathfinder was directing it to the same way after Flee ended.
To solve this, I expanded pathfinder to register the dangerous spots and avoid them when calculating the path.
Still, the Escaper sometimes got stuck in that state even though it happened less frequently.
Turned out, while the pathfinder found another way, it still involved moving near the Finder.
So I make the Escaper more "brave" by making it only flee when the Finder is going to see it.
Now, it can even walk behind the Finder's back.



\section{Reflection}
\subsection{Results and Behavior}
Discuss the outcomes of your implementation:
\begin{itemize}
    \item How well the RTS unit behaved according to expectations.
    \item Examples of scenarios where the AI worked as intended.
\end{itemize}

\subsection{Issues and Limitations}
Highlight challenges faced during implementation, such as:
\begin{itemize}
    \item Edge cases (e.g., units getting stuck during pathfinding).
    \item Limitations of the FSM model or the pathfinding system.
\end{itemize}

\subsection{Alternative Solutions and Improvements}
Suggest possible improvements or alternative techniques, referencing research or practical solutions. Discuss future directions for enhancing unit AI.

\section{Conclusion}
\subsection{Summary}
Summarize the contributions of the report, including insights into the techniques and their implementation.

\subsection{Critical Reflection}
Reflect on your individual performance in the project:
\begin{itemize}
    \item What went well.
    \item Areas for improvement.
\end{itemize}

\section*{References}
List all references used in the report in a standard format (e.g., IEEE, APA).

\section*{Appendices (Optional)}
\subsection*{Code Snippets}
Include any relevant code snippets or detailed diagrams.

\end{document}
